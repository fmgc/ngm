\documentclass[tikz]{standalone}
\usepackage[utf8x]{inputenc}

%
%
% SUBFIGURE
%
%
\usepackage{subfigure}
\subfigtopskip=0pt
\subfigcapskip=0pt
\subfigbottomskip=0pt
%
%
%
%	TIKZ 
%
\usepackage{tikz}
\usetikzlibrary{calc}
\usetikzlibrary{backgrounds}
\usetikzlibrary{patterns}
\usetikzlibrary{intersections} 
\tikzset{obs/.style={var, fill = black!10}}
\tikzset{rel/.style={fill = black!5}}

%
%	For influence diagrams
\usetikzlibrary{shapes.misc}
%
% common style
\tikzset{idnode/.style={inner ysep = 4pt,inner xsep = 8pt, draw}}
%
% random var
\tikzset{varnode/.style={idnode,rounded rectangle}}
%
% random var
\tikzset{detnode/.style={idnode,rounded rectangle,double}}
%
% utility
\tikzset{utilnode/.style={idnode, chamfered rectangle, chamfered rectangle 
		xsep=32pt, chamfered rectangle ysep=0pt}}
%
% action
\tikzset{actionnode/.style={idnode,rectangle}}
%
% arrows
\tikzset{cond/.style={>=stealth,draw}}
%\tikzset{cond/.style={>=latex,draw}}

\tikzset{var/.style={draw, rounded rectangle}}
\tikzset{det/.style={double}}
\tikzset{action/.style={draw,rectangle}}

%
%	PGFPLOTS
\usepackage{pgfplots}
\pgfplotsset{compat=1.9}
\usepgfplotslibrary{statistics,fillbetween}
\pgfdeclarelayer{background}
\pgfdeclarelayer{foreground}
\pgfsetlayers{background,main,foreground}
%

%
%	ACTIONS
%
\newcommand{\aup}{\ensuremath{\uparrow}}
\newcommand{\adown}{\ensuremath{\downarrow}}
\newcommand{\aleft}{\ensuremath{\leftarrow}}
\newcommand{\aright}{\ensuremath{\rightarrow}}
\newcommand{\apick}{\ensuremath{\Uparrow}}
\newcommand{\adrop}{\ensuremath{\Downarrow}}
\newcommand{\askip}{\ensuremath{\emptyset}}


%
%	PERCEPT VALUES
%
\newcommand{\pempty}{\ensuremath{\bigcirc}}
\newcommand{\pgold}{\ensuremath{\otimes}}
\newcommand{\pminer}{\ensuremath{\odot}}
\newcommand{\pobstacle}{\ensuremath{\oplus}}

%
% LOCAL TOPO DIAGRAMS
%

\newcommand{\loccorner}[1] {
		\node[var, rel] (C0) at (1,1) {};
		\node[var] (C1) at (2,1) {};
		\node[var] (C2) at (1,0) {};
		\node[var] at (0,1) {};
		\node[var] at (1,2) {};
		\node at (2, 0) {#1};
		\path
			(C1) edge (C0)
			(C2) edge (C0)
		;
}

\newcommand{\locmiddle}[1] {
		\node[var, rel] (M0) at (1,1) {};
		\node[var] (M1) at (2,1) {};
		\node[var] (M2) at (1,0) {};
		\node[var] (M3) at (0,1) {};
		\node[var] (M4) at (1,2) {};
		\node at (2, 0) {#1};
		\path
			(M1) edge (M0)
			(M2) edge (M0)
			(M3) edge (M0)
		;				
}

\newcommand{\loccenter}[1] {
		\node[var, rel] (X0) at (1,1) {};
		\node[var] (X1) at (2,1) {};
		\node[var] (X2) at (1,0) {};
		\node[var] (X3) at (0,1) {};
		\node[var] (X4) at (1,2) {};
		\node at (2, 0) {#1};
		\path
			(X1) edge (X0)
			(X2) edge (X0)
			(X3) edge (X0)
			(X4) edge (X0)
		;				
}

\newcommand{\sensorgrid}{
		\node[var,obs] (S0) at (0,2) {$Y_0$};
		\node[var,obs] (S1) at (1,2) {$Y_1$};
		\node[var,obs] (S2) at (2,2) {$Y_2$};
		\node[var,obs] (S3) at (0,1) {$Y_3$};
		\node[var,obs] (S4) at (1,1) {$Y_4$};
		\node[var,obs] (S5) at (2,1) {$Y_5$};
		\node[var,obs] (S6) at (0,0) {$Y_6$};
		\node[var,obs] (S7) at (1,0) {$Y_7$};
		\node[var,obs] (S8) at (2,0) {$Y_8$};

		\path[dotted]
			(S0) edge (S1)
			(S1) edge (S2)
			(S3) edge (S4)
			(S4) edge (S5)
			(S6) edge (S7)
			(S7) edge (S8)
			(S0) edge (S3)
			(S1) edge (S4)
			(S2) edge (S5)
			(S3) edge (S6)
			(S4) edge (S7)
			(S5) edge (S8)
		;				
}

%
% MATH
%
\newcommand{\at}[1]{\ensuremath{\!\left(#1\right)}}
\newcommand{\PARS}[1]{\left\{ #1 \right\}}
\newcommand{\SET}[1]{\PARS{#1}}
\newcommand{\PC}[2] {\ensuremath{P\at{#1\middle| #2}}}
\newcommand{\col}[1] {\ensuremath{\begin{array}[c]#1\end{array}}}

%
%
%	ACRONYMS
%
\usepackage{acronym}
%
\acrodef{BRF}[BRF]{belief-revision function}
\acrodef{PCF}[PCF]{percept-correction function}
\acrodef{CPT}[CPT]{conditional probability table}
\acrodef{PGM}{Probabilistic Graphical Model}
\acrodef{ML}{Machine Learning}
\acrodef{BDI}{Beliefs, Desires and Intentions}
\acrodef{SRL}{Statistical Relational Learning}
\acrodef{PLP}{Probabilistic Logic Programming}
\acrodef{ASL}[\textsc{ASL}]{\textsc{AgentSpeak(L)}}
\acrodef{BN}{Bayesian Network}
\acrodef{DBN}{Dynamic Bayesian Network}
\acrodef{HMM}{Hidden Markov Model}
\acrodef{AI}{Artificial Intelligence}
\acrodef{PCA}{Principal Component Analysis}
\acrodef{DNN}{Deep Neural Networks}
\acrodef{CPD}{Conditional Probability Distribution}
\acrodef{MEU}{Maximum Expected Utility}
\acrodef{MAP}{Maximum \emph{a posteriori}}
\acrodef{MkL}{Markov Logic}
\acrodef{PPC}{Probabilistic Preception Correction}
%

%
%	COMMANDS
%
\newcommand{\JASON}{\textsc{Jason}}
\newcommand{\DLIB}{\textsc{Dlib}}
\newcommand{\GM}{\textsc{GoldMiners}}
\newcommand{\WEKA}{\textsc{weka}}
\newcommand{\SAMIAM}{\textsc{samiam}}
\newcommand{\PROLOG}{\textsc{ProLog}}
\newcommand{\JAVA}{\textsc{Java}}
\newcommand{\GALAXITY}{\textsc{Galaxity}}
\newcommand{\OPENMARKOV}{{\textsc{OpenMarkov}}}

\newcommand{\Sel}[1]{\ensuremath{\cal{S}_{\cal{#1}}}}
\newcommand{\SE}{\ensuremath{\textrm{selectEvent}}}
\newcommand{\SO}{\ensuremath{\textrm{selectOption}}}
\newcommand{\SI}{\ensuremath{\textrm{selectIntention}}}
\newcommand{\AUTHNOTE}[1]{{\color{blue} [#1]}}
\newcommand{\GOLDS}{\ensuremath{\mathbf{G}}}
\newcommand{\mathtxt}[1]{\ensuremath{\textrm{#1}}}
\newcommand{\pr}{\ensuremath{\textrm{P}}}
\newcommand{\PR}[1]{\ensuremath{\pr\at{#1}}}
\newcommand{\GIVEN}{\ensuremath{~\middle|~}}
\newcommand{\AT}[1]{\!\left(#1\right)}
\newcommand{\TS}[1]{\ensuremath{^{(#1)}}}
\newcommand{\INDG}[3]{\ensuremath{#1~\perp~#2~\mid~#3}}
%

% %

\begin{document}
	\begin{tikzpicture}	
		\begin{axis}[smooth,
			width = \textwidth,
			height = \textwidth,
			title = {Effect of Perception Correction on Processing Time},
			xlabel = {Sensor noise}, ylabel = {Processing Time (ms)},
			xmin = -0.005, xmax = 0.205,
			xtick = {0.0, 0.025, 0.05, 0.075, 0.1, 0.125, 0.15, 0.175, 0.2},
			xticklabel style = {
				/pgf/number format/precision = 3,
				/pgf/number format/fixed,
				/pgf/number format/fixed zerofill,
			},
			mark options = {scale = 2,},
			]
			%%%%%%%%%%%%%%%%%%%%%%%%%%%%%%%%%%%%
			%
			%	MEAN PLOTS
			%
			%	dummy
			%
			%
			%	noc
			%
			\addplot[solid,black,mark=square]
				table[ col sep = comma,
					x = noise,
					y = timem ] {nopc_summary.csv};
			\addlegendentry{smart};
			% %
			% %	cor
			% %
			\addplot[solid,black,mark=o]
				table[ col sep = comma,
					x = noise,
					y = timem ] {pc_summary.csv};
			\addlegendentry{corrected};
			% %
			% %%%%%%%%%%%%%%%%%%%%%%%%%%%%%%%%%%%%
			% %
			% %	MEAN +/- STDVAR PLOTS
			% %
			% %
			% %	dummy
			% %
			% \addplot[black!10, name path = DMIN]
			% 	table[x=NOISE,y=DUMMYMIN] {experiment2.tex};	
			% \addplot[black!10, name path = DMAX]
			% 	table[x=NOISE,y=DUMMYMAX] {experiment2.tex};
			% %
			% %	noc
			% %
			\addplot[black!10, name path = NMIN]
				table[ col sep = comma,
					x = noise,
					y expr = \thisrow{timem} - 0.5*\thisrow{times} ] {nopc_summary.csv};
			\addplot[black!10, name path = NMAX]
				table[ col sep = comma,
					x = noise,
					y expr = \thisrow{timem} + 0.5*\thisrow{times}] {nopc_summary.csv};
			% %
			% %	cor
			% %
			\addplot[black!10, name path = CMIN]
				table[ col sep = comma,
					x = noise,
					y expr = \thisrow{timem} - 0.5*\thisrow{times} ] {pc_summary.csv};
			\addplot[black!10, name path = CMAX]
				table[ col sep = comma,
					x = noise,
					y expr = \thisrow{timem} + 0.5*\thisrow{times}] {pc_summary.csv};
			%
			%%%%%%%%%%%%%%%%%%%%%%%%%%%%%%%%%%%%
			%
			%	FILL PLOTS
			%
			%
			%	dummy
			%
			%\addplot[black!10, opacity=0.25]
			%	fill between [of= DMIN and DMAX];
			%
			%	noc
			%
			\addplot[black!10, opacity=0.25]
				fill between [of= NMIN and NMAX];
			%
			%	cor
			%
			\addplot[black!20, opacity = 0.25]
				fill between [of= CMIN and CMAX];
		\end{axis}
	\end{tikzpicture}
	\begin{tikzpicture}
		\begin{axis}[smooth,
			width = \textwidth,
			height = \textwidth,
			title = {Effect of Perception Correction on Gathered Golds},
			xlabel = {Sensor noise}, ylabel = {Gathered golds},
			xmin = -0.005, xmax = 0.205,
			xtick = {0.0, 0.025, 0.05, 0.075, 0.1, 0.125, 0.15, 0.175, 0.2},
			xticklabel style = {
				/pgf/number format/precision = 3,
				/pgf/number format/fixed,
				/pgf/number format/fixed zerofill,
			},
			mark options = {scale = 2,},
			]
			%%%%%%%%%%%%%%%%%%%%%%%%%%%%%%%%%%%%
			%
			%	MEAN PLOTS
			%
			%	dummy
			%
			\addplot[solid,black,mark=diamond]
				table[ col sep = comma,
					x = noise,
					y = goldsm ] {dummy_summary.csv};
			\addlegendentry{dummy};
			%
			%	noc
			%
			\addplot[solid,black,mark=square]
				table[ col sep = comma,
					x = noise,
					y = goldsm ] {nopc_summary.csv};
			\addlegendentry{smart};
			% %
			% %	cor
			% %
			\addplot[solid,black,mark=o]
				table[ col sep = comma,
					x = noise,
					y = goldsm ] {pc_summary.csv};
			\addlegendentry{corrected};
			% %
			% %%%%%%%%%%%%%%%%%%%%%%%%%%%%%%%%%%%%
			% %
			% %	MEAN +/- STDVAR PLOTS
			% %
			% %
			% %	dummy
			% %
			\addplot[black!10, name path = DMIN]
				table[ col sep = comma,
					x = noise,
					y expr = \thisrow{goldsm} - 0.5*\thisrow{goldss} ] {dummy_summary.csv};
			\addplot[black!10, name path = DMAX]
				table[ col sep = comma,
					x = noise,
					y expr = \thisrow{goldsm} + 0.5*\thisrow{goldss}] {dummy_summary.csv};
			% %
			% %	noc
			% %
			\addplot[black!10, name path = NMIN]
				table[ col sep = comma,
					x = noise,
					y expr = \thisrow{goldsm} - 0.5*\thisrow{goldss} ] {nopc_summary.csv};
			\addplot[black!10, name path = NMAX]
				table[ col sep = comma,
					x = noise,
					y expr = \thisrow{goldsm} + 0.5*\thisrow{goldss}] {nopc_summary.csv};
			% %
			% %	cor
			% %
			\addplot[black!10, name path = CMIN]
				table[ col sep = comma,
					x = noise,
					y expr = \thisrow{goldsm} - 0.5*\thisrow{goldss} ] {pc_summary.csv};
			\addplot[black!10, name path = CMAX]
				table[ col sep = comma,
					x = noise,
					y expr = \thisrow{goldsm} + 0.5*\thisrow{goldss}] {pc_summary.csv};
			%
			%%%%%%%%%%%%%%%%%%%%%%%%%%%%%%%%%%%%
			%
			%	FILL PLOTS
			%
			%
			%	dummy
			%
			\addplot[black!10, opacity=0.25]
				fill between [of= DMIN and DMAX];
			%
			%	noc
			%
			\addplot[black!10, opacity=0.25]
				fill between [of= NMIN and NMAX];
			%
			%	cor
			%
			\addplot[black!20, opacity = 0.25]
				fill between [of= CMIN and CMAX];
		\end{axis}
	\end{tikzpicture}	
\end{document}